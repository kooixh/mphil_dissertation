%% 
%% ACS project dissertation template. 
%% 
%% Currently designed for printing two-sided, but if you prefer to 
%% print single-sided just remove ",twoside,openright" from the 
%% \documentclass[] line below. 
%%
%%
%%   SMH, May 2010. 


\documentclass[a4paper,12pt]{report}


%%
%% EDIT THE BELOW TO CUSTOMIZE
%%

\def\authorname{Xiu Hong\ Kooi\xspace}
\def\authorcollege{Wolfson College\xspace}
\def\authoremail{xhk20@cam.ac.uk}
\def\dissertationtitle{Exploring Dependent Types and Behaviour in Imperative Language}
\def\wordcount{0}


%\usepackage[dvips]{epsfig,graphics} 
\usepackage{epsfig,graphicx,verbatim,parskip,tabularx,setspace,xspace}
\usepackage{amsfonts}
\usepackage{amsmath}
\usepackage{amssymb}
\usepackage{listings}
\usepackage{float}

\usepackage[british]{babel}
\usepackage[%
  backend=bibtex      % biber or bibtex
%,style=authoryear    % Alphabeticalsch
 ,style=numeric-comp  % numerical-compressed
 ,sorting=none        % no sorting
 ,sortcites=true      % some other example options ...
 ,maxbibnames=99
 ,block=none
 ,indexing=false
 ,citereset=none
 ,isbn=true
 ,url=true
 ,doi=true            % prints doi
 ,natbib=true         % if you need natbib functions
]{biblatex}
\usepackage{biblatex}
\addbibresource{./dissertation.bib}


%% START OF DOCUMENT
\begin{document}


%% FRONTMATTER (TITLE PAGE, DECLARATION, ABSTRACT, ETC) 
\pagestyle{empty}
\singlespacing
% title page information
\begin{titlepage} 

\begin{center}
\noindent
\huge
\dissertationtitle \\
\vspace*{\stretch{1}}
\end{center}

\begin{center}
\noindent
\huge
\authorname \\
\Large
\authorcollege      \\[24pt]
\mbox{}\\
%\begin{figure}
\includegraphics{CUni3.pdf}
%\end{figure}
\end{center}

\vspace{24pt} 

\begin{center}
\noindent
\large
{\it A dissertation submitted to the University of Cambridge \\ 
in partial fulfilment of the requirements for the degree of \\ 
Master of Philosophy in Advanced Computer Science} 
\vspace*{\stretch{1}}
\end{center}

\begin{center}
\noindent
University of Cambridge \\
Computer Laboratory     \\
William Gates Building  \\
15 JJ Thomson Avenue    \\
Cambridge CB3 0FD       \\
{\sc United Kingdom}    \\
\end{center}

\begin{center}
\noindent
Email: \authoremail \\
\end{center}

\begin{center}
\noindent
\today
\end{center}

\end{titlepage} 

\newpage
\vspace*{\fill}

\onehalfspacing
\newpage
{\Huge \bf Declaration}

\vspace{24pt} 

I \authorname of \authorcollege, being a candidate for the M.Phil in
Advanced Computer Science, hereby declare that this report and the
work described in it are my own work, unaided except as may be
specified below, and that the report does not contain material that
has already been used to any substantial extent for a comparable
purpose.

\vspace{24pt}
Total word count: \wordcount

\vspace{60pt}
\textbf{Signed}: 

\vspace{12pt}
\textbf{Date}:


\vfill

This dissertation is copyright \copyright 2020 \authorname. 
\\
All trademarks used in this dissertation are hereby acknowledged.



\newpage
\vspace*{\fill}

\singlespacing
\newpage
{\Huge \bf Abstract}
\vspace{24pt} 


Write a summary of the whole thing. Make 
sure it fits in one page. 


\newpage
\vspace*{\fill}


\pagenumbering{roman}
\setcounter{page}{0}
\pagestyle{plain}
\tableofcontents
\listoffigures
\listoftables

\onehalfspacing

%% START OF MAIN TEXT 

\chapter{Introduction}
\pagenumbering{arabic} 
\setcounter{page}{1} 
\textit{Type systems} \cite{typesystem} have been one of the most extensively researched field in 
Programming Languages. They act as a way from improving the reliability of a 
language by enforcing rules, preventing operations being applied on 
incompatible data. Type systems can be broken down into multiple categories but 
two of the most well known are \textit{Static} \cite{staticTyping} and 
\textit{Dynamic} \cite{dynamicTyping} typing. Mainstream programming 
languages such as \textit{Java} \cite{java}, \textit{C} \cite{c} and \textit{C++} \cite{cpp} 
uses the former while languages like \textit{Python} \cite{python} and 
\textit{JavaScript} \cite{js} uses the latter. 
Over the years, programming languages have included more powerful and flexible 
type systems, languages like \textit{C\#} \cite{cSharp} and \textit{Go} \cite{goInferenceType} allow 
\textit{type inference} \cite{inferenceType}, using a feature called \textit{Reflection} 
Java and Python can even achieve \textit{Duck Typing} \cite{javaDuckType}.

\par
However, as much as we have studied about type systems, \textit{Dependent Types} 
\cite{depenTypeAtWork} remains uncommon in the industry. While the theory of dependent types has been 
established several decades ago, only a small number of languages has 
integrate full dependent type support, most of them being functional languages. 
Dependent types allows the programmer to create types whose definition depends 
on a value. A type system that provides such refined control over the values it 
can take unlocks possibility that are previously unavailable such as 
domain-specific type checking at compile time. Furthermore, it acts as a built 
in ``error-handling'' code and could potentially 
reduces the lines of code a programmer needs to write. An in depth 
definition of dependent types is provided in the next section.

\par
The expressive nature of dependent types allow one to define complex 
mathematical assertations and hence lends itself to theorem proving systems. 
Mutliple functional languages such as \textit{Epigram} 
\cite{epigram} and \textit{Agda} \cite{agda} has 
built in support for dependent typing. However they remain niche and more 
mainstream languages like Java and C++ does not get the luxury. In chapter 2 of 
this dissertation we will be providing an in depth analysis of the current state 
of dependent types in programming languages and answer why dependent types are 
not more prominant in languages, in particular imperative languages.

\par
Multiple past research has studied and show the feasibility of an dependently typed 
imperative language. These studies has expressed the semantics of 
how an imperative languages would interact with dependent types and in some 
cases created a new language from the ground up. While we find these work novel 
and provided highly technical explanation to the field, they fail to caputure 
how this could relate to mainsteam programming languages. Furhtermore, the 
semantics provided in these studies are very complex with advanced language 
features, we propose that a more simple and barebone definiton will allow easier 
access to the literature and motivate more studies into dependenty types. In 
chapter 3 we will discuss all the work done in the area and point out some of the 
motivation for this project.

\par
In chapter 4 we will be defining the syntax, typing rules and semantics 
of a basic language with dependent types support which will be studied 
and extended throughout the disseratation. In chapter 5 we will be observing 
multiple scenarios that dependent types cause uncertainties when used with 
imperative languages. We will be providing the semantics required for them 
to behave correctly.

\par
Finally we will conclude by discussing advanced features that can be integrated 
into our languages and how it can further complicate the language. We will also 
discuss some of the other considerations that need to be taken account when 
thinking about incorporating dependent types into a real language. 

\section{What are dependent types}
In this section we will be providing the definitions of dependent types.

\subsection{Basic Definition}
At a very high level dependent types are types that depends on the value of 
another type. For example, we can define a type that captures only the even 
integers using the definition 
\verb+type EvenInt := { i : Int | i % 2 = 0}+. In this case we 
can say that the type EvenInt \textit{EvenInt} depends on the type \textit{int}.
Another commonly used example to describe dependent types would be a type like 
\verb+type FixVec<T, N> := { v : Vec<T> | len(v) = N}+, this type definition defines 
a vector or array of elements with type $T$ that always contains $N$ elements.

\subsection{Dependent $\Pi$ Types}
We can capture the definition mathematically using the notion of \textit{dependent 
product types}, i.e. $\Pi$ type. This is also sometimes referred to as 
\textit{dependent function type} as in this definition we construct a function 
$F: A \rightarrow B$. The function $F$ takes an element of type $A$ and 
gives us an element of type $B$ which depends on $A$. We express it 
mathematically using the $\Pi$ notation as
\begin{center}
 \begin{tabular}{l}
   $\prod x: A.  F(x)$
 \end{tabular} 
\end{center}

In this definition, $F(x)$ is the type family for the type $B$ that depends on $A$.
However $F$ could be a constant function, so we can also express the definition 
as $\Pi x:A.B$, in this case $B$ does not depend 
on the value $x$. Using the \textit{EvenInt} example from earlier, 
it can be defined as 
$\Pi x:Int.\text{ }\{ i:Int\text{ }|\text{ }i\text{ }\%\text{ }2\text{ }= 0\}$.

\par
Interestingly, the dependent product type correspond to the 
\textit{forall quantifier} as per 
the \textit{Curry–Howard correspondence}. The idea is that the dependent 
function $F(x)$ correspond to predicate $P(x)$ and thus the dependent product 
type has a one-to-one correspondence to $\forall x: A. P(x)$.

\subsection{Dependent $\Sigma$ Types}
In addition to the dependent product type, we have the notion of \textit{dependent sum 
types}, written as $\Sigma$ type. This is often referred to as the 
\textit{dependent pair type} as the resulting type here is an ordered pair. 
Specifically the resulting pair $\langle a,b \rangle$ is ordered such that the 
second element depends on the first element. The 
mathematical definition is similar to that of the product type
\begin{center}
 \begin{tabular}{l}
   $\langle a,b \rangle :\sum x: A.  F(x)$
 \end{tabular} 
\end{center}
In the case $a:A, b: F(x)$, similarly, $F$ could be a constant function and thus 
the expression is $\Sigma x:A.B$. Consider the following example, 

$\Sigma x: Int.\{y:Int\text{ }|\text{ } y = x * 2\}$, then the type would 
contain values like $\langle 1,2 \rangle$ and $\langle 4,8 \rangle$ where the 
second pair is doubled the first.

\par
Like the dependent product type, the dependent sum type correspond to a 
universal quantifier, in this case, the \textit{existential quantifier}. As 
per the Curry–Howard correspondence, $F(x)$ corresponds to predicate $P(x)$ 
thus $\Sigma x:A.F(x)$ correponds to $\exists x: A. P(x)$.

\par
While both dependent product types and dependent sum types are important to the 
literature, the project itself will mainly focus on the former. we believe that 
the the notion of pair in dependent sum types prove to be redundant in 
the construction a programming language and does not provide any additional 
value. The dependent product type is largely adequate for our goal.

\chapter{State of Dependent Types} 
In this chapter we will be reviewing the current knowledge of dependent types 
in different programming languages. It will cover languages with full dependent 
types support as well as some languages with similar concepts and point out how 
it differs from dependent types. Lastly we will summarise all these languages 
and point out some limitations and unknowns. 

\section{Functional Languages}

\subsection{Agda}

\textit{Agda} \cite{agda} is a purely functional language originally developed by Ulf Norell in 
1999 however the first appearance the current version known as Agda 2 is in 
2007. Agda has all the necessary constructs one would expect in a functional 
language such as first-class functions, inductive definitions, pattern matching, 
etc. In addition to being a functional language, Agda also serves an automated theorem prover. 
Agda is one of the few programming language with native dependent type support. 

\par
The code listing below is an example of defining a fixed length vector in 
dependent type. 

\begin{figure}[H]
  \begin{lstlisting}  
  data Nat : Set where
  zero : Nat
  suc  : Nat -> Nat  
  
  data Vec (A : Set) : Nat -> Set where
  [] : Vec A zero
  _::_ : {n : Nat} -> A -> Vec A n -> Vec A (suc n)
  \end{lstlisting}
  \caption{Dependent Types in Agda}
\end{figure}

\par
While Agda provides dependent type support, it remains a niche language. One of 
the reason being its paradigm. Agda is \textit{purely functional} \cite{purelyFP}, meaning that 
all functions are pure (i.e. not relying on the program state or other mutable 
data). Functional languages are generally considered harder to learn and grasp 
compared to other paradigms \cite{fpHarder}. For that reason Agda remain niche in 
the programming world and is predominantly used for theorem proving.

\subsection{Idris}

\chapter{Related Work}

\chapter{The Basic Language}

\chapter{The Semantics}

\chapter{Further Work}

\chapter{Summary and Conclusions} 


\appendix
\singlespacing

\printbibliography

\end{document}
