\newpage
{\Huge \bf Abstract}
\vspace{24pt} 

Refinement Types provide a more expressive type checking and allow more errors to be caught automatically, however it is not a feature the general programmer is aware of. Existing research in the area is focused predominantly on the theoretical aspects of refinement types rather than their implementation in real-world programming. While there are certain programming languages available that support refinement types, these languages are often limited to a small number of research projects that are no longer maintained or simply open-source libraries that simulate a restricted form of refinement types. The introduction of refinement types in a mainstream programming proves to be difficult because of features like mutation. Refinement types are able to make use of free variables in the program, this means that a change in the program state might affect the meaning of types, making type checking ambiguous. Furthermore, in general showing that a predicate holds is undecidable, making static type checking of refinement types hard. Our project aims to present refinement types with an emphasis on their behaviour in a real-world programming language. We achieve this by describing an imperative language named Simple-$R$ which strongly resembles C. We address the problems stated by adapting multiple well-known concepts in programming languages such as immutable variables, closures and hybrid type checking. Using a combination of these techniques we formalise the rules for variables and pointers in Simple-$R$ in order to specify the behaviour of refinement types precisely. 

\newpage
\vspace*{\fill}
